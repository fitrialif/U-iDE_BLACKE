\documentclass{article}

\usepackage[final]{nips_2018}

% to avoid loading the natbib package, add option nonatbib:
% \usepackage[nonatbib]{nips_2018}

\usepackage[utf8]{inputenc} % allow utf-8 input
\usepackage[T1]{fontenc}    % use 8-bit T1 fonts
\usepackage{hyperref}       % hyperlinks
\usepackage{url}            % simple URL typesetting
\usepackage{booktabs}       % professional-quality tables
\usepackage{amsfonts}       % blackboard math symbols
\usepackage{nicefrac}       % compact symbols for 1/2, etc.
\usepackage{microtype}      % microtypography

\usepackage{graphicx}
\usepackage{cite}

\title{BLACKE: Body Language Affect Classification by Keypoint Estimation}

\author{
  Undergrad-ient Descent Expedition: \\
  Jack (Xiang) Zhou, Gurkaran Aujla, James Bie, Eric Gao, Insoo Rhee\\
  Faculty of Applied Sciences\\
  Simon Fraser University\\
  Burnaby, Canada\\
  \texttt{xza194@sfu.ca} \\
}

\begin{document}

\maketitle

\begin{abstract}
Current state of the art development in affective classification of people on visual modalities usually focuses on facial expression recognition (FER) but it has some shortcomings in its real world usage. The present work tackles the affective classification problem in body language by using keypoint estimation as a pre-processing step as opposed to previously used convolutional methods. Na\"ive neural network classifiers are trained to test for the plausibility of the approach and the approach was found to be plausible. The present approach implies a significant dimensionality reduction in classifiers needed to solve this problem.
\end{abstract}

\section{Introduction}

Affective Computing is a rapidly growing interdisciplinary field that is concerned with interpretations and implications of human emotion in computation. It takes findings and motivations from computing science and psychology into account to piece together approaches to solve problems related to human affect. One of the most fundamental problems in Affective Computing is the classification of an individual's emotional state given some information about such individual on particular modalities such as text, auditory, behavioural, and visual.

For the visual modality, the present state of the art approach to addressing this problem is by applying Facial Expression Recognition (FER). The process involves training classifiers that first identifies faces and then taking the images of faces as a means to identify an individual's emotional state through another trained classifier \citep{li2018deep}. The overall method is successful and current approaches with deep neural networks and such can achieve near perfect accuracy ratings on testing data. 

While FER is a successful method, it is not applicable when the face of the individual is not seen or covered by objects or body limbs. Observing facial information is often key and a dominant factor when it comes to deciding the current emotional state of a person. The absence of this information would imply a large increase in variability in evaluating the emotional state of a person, but in order to answer the question of "what is this person feeling", we have to look at tackling the problem with other available information.

\subsection{Problem Statement}

In the present work we want to address this shortcoming by identifying the emotional state of an individual in terms of only the body language that they are exhibiting (ie. their posture). There are still associations linked with understanding the emotional state of a person when just observing their body language. For example, hands covering face is usually associated with sadness or distress, and extended arms raised high up is usually associated with happiness or surprise. However, this is inherently a difficult problem for humans, in particular with respect to some emotional states over others \citep{schindler2008recognizing}\citep{de2011bodily}. 

Previous work has been done on the topic using a neurologically inspired convolutional model that takes in visual data containing both body language and facial expressions \citep{schindler2008recognizing}.

\begin{figure}[h]
	\centering
	\includegraphics[scale=0.3]{schindlernet}
	\caption{Model used in Schindler et al. 2008}
\end{figure}

As seen in Figure 1, the model takes in an input image and applies orientation filtering, pooling, and template filtering on the image before feeding the filtered data into a classifier to arrive at a category decision. It is noteworthy to see that the steps taken to filter and pre-process the image simulates corresponding to various areas of the human visual cortex, going from low level (ex. V1 area) information processing progressively to higher level information processing (ex. V4/IT area).

\subsection{Proposed Work}

While the convolutional filtering approach is plausible and able to perform just as well or better than human test subjects, the model is prone to accounting for extraneous information in its processing steps. The dataset that was used for their experimental work were custom constructed and standardized on neutral backgrounds so the introduction of non-neutral backgrounds (ie. in real life) may be a highly threatening confound.

The present work wishes to approach the same problem by using a different approach to pre-process the input data to a high level representation that primarily accounts for background noise as well as providing other features. Namely, using a representation of the pose of the body in the input image.

The underlying motivation in play here is as follows: Suppose one is visually observing a person, knowing about only the pose that the person is making preserves the relevant information regarding the person's emotional state (Figure 2).

\begin{figure}[h]
	\centering
	\includegraphics[scale=0.4]{kran}
	\caption{The extracted keypoints of the person's body and face contain the same amount of semantic information regarding the person's emotional state as the full image. Image courtesy of one of our authors Gurkaran Aujla}
\end{figure}

We hypothesize that using a presentation of body posture is a feasible alternative to convolutional pre-processing for classifying emotional states based on just body language. The confirmation of this hypothesis would imply a significant dimensionality reduction to the input data needed for a classifier model, and that leads to more compact models and faster training times along with other implications.

\subsection{Keypoint Estimation}

To achieve the goal of representing a pose in a body, we look to applying keypoint estimation with OpenPose \citep{cao2017realtime} \citep{simon2017hand} \citep{wei2016cpm} created by the CMU Perceptual Computing Laboratory, which is open source and freely available for non-commercial use. 

OpenPose is a system that allows for real-time keypoint estimation and detection of human bodies in images and videos. It is capable of finding estimated locations of several body keypoints in the body, hands, and face. It is a capable of doing these simultaneously with multiple people as well. 

Openposes’s	 basic algorithm finds each type of keypoint in all bodies simultaneously, and then finds the connection between these keypoints in a separate neural network. The program can take most image and video formats as inputs, and it will output a variety of formats depending on the configurations. For example, it can output an image with the keypoints overlaid on the people in the image (ex. Figure 2), it can also output JSON files for the keypoints. In the JSON file, it contains the $(x,y)$ coordinates of every bone detected, as well as a confidence rating $c$ for the position of each keypoint expressed as a percentage.

\subsection{Viewpoint Invariance}

As with all work done in computational vision, it is hard to know the exact shape, size and distance of an object in an image as it is only a projection of the original object. The problem of extracting this type of information is called the inverse projection problem, and it is a non-trivial problem.

The ability to account for the inverse projection problem and ultimately achieving viewpoint invariance in models has been a difficult task, especially when the only given information is just a two dimensional image most of the time. The present direction in accounting for this problem is by constructing 3D representations of the object being seen in the image (for example refer to \citep{shin2018pixels} ). The present work faces many challenges related to this problem and they will be addressed separately in the coming sections.

\section{Approach}

In this section we describe various aspects and considerations to setting up our models and overall approach.

\subsection{Data}

\subsubsection{Dataset Composition}
Constructing a dataset was a non-trivial task as it is difficult to find labelled datasets that have faces, full body, and is labelled in a uniform manner. Previous work in this field involved manual construction of data from actors \citep{schindler2008recognizing}. For the current work we are piecing together a dataset from various sources. However, since only the posture information will remain, the colorization and image dimensions do not matter.

We used the BEAST dataset \citep{de2011bodily} from the Brain and Emotion Laboratory in Maastricht University which had labelled grayscale images of individuals with their face censored. Images in this dataset are categorized into four emotional categories: Happy, Sad, Angry, and Fearful. We also obtained another dataset from the same laboratory which had the same type of data but colored and larger to include in the dataset that we are piecing together \citep{stienen2012computational}.

From there onwards, the team found suitable images of people online by various means such as image search engines and news articles. The sampled images are then hand-labelled. In order to generate more datapoints, we applied jittering techniques and reflections to transform the data that we already have.

\subsubsection{Pre-processing}

To prepare any piece of data for the classifiers that we are going to train, the input image has to be fed through OpenPose first to extract a skeleton from the image. One particular advantage to using the Keypoint Estimation approach is that the input dimensions of the image and format do not matter; the output is invariant regardless of these factors. We chose to use the BODY\_25 model which contains 25 body keypoints, as shown in Table 1.

\begin{table}[h]
	\caption{Body Keypoint numbers and corresponding body parts in BODY\_25}
	\centering
	\begin{tabular}{cccccccc}
	\toprule
	0 & 1 & 2 & 3 & 4 & 5 & 6 \\
	Nose & Neck & R\_Shoulder & R\_Elbow & R\_Wrist & L\_Shoulder & L\_Elbow\\	
	\midrule
	8 & 9 & 10 & 11 & 12 & 13 & 14\\
	L\_Wrist & MidHip & R\_Hip & R\_Knee & R\_Ankle & L\_Hip & L\_Knee\\
	\midrule
	15 & 16 & 17 & 18 & 19 & 20 & 21\\
	L\_Ankle & R\_Eye & L\_Eye & R\_Ear & L\_Ear & L\_BigToe & L\_SmallToe\\
	\midrule
	22 & 23 & 24 & 25\\
	L\_Heel &R\_BigToe & R\_SmallToe & R\_Heel\\
	\midrule
	\end{tabular}
\end{table}

Each body keypoint that is extracted from the image comes with a 3-tuple of values $(x,y,c)$ denoting the normalized coordinates (in percentages) of the keypoint and the confidence in the measurement in this point. The total comes to 75 numbers per body, which is significantly less data than an image.

\subsection{Invariance}
Depth perception is a major issue in the present work that has to be addressed. Objects that are closer appear larger than object that are farther away. Humans have a way to account for this issue by preconceived prior knowledge about the world as well as exploiting retinal disparity. However, judging from solely the keypoint information, it is hard to tell whether if someone is tall or if someone is close. The overall translation of the person will also have an impact on what the classifier receives, as the overall position of a person in an image have no bearing on the emotional state of the person.

We propose a transformation over the estimated keypoints that ultimately results in an invariant form that accounts for some of these issues.

\subsubsection{Invariant Form Transformation}

To begin, the immediate goal towards a fitting transformation should be that it is translation independent, where any arbitrary set of keypoints should be semantically equivalent to a translated version of those keypoints. 

The secondary goal is to discard information regarding the length of the bones in the set of keypoint to account for depth perception.

The present approach takes advantage of the fact that the connections between the keypoints in a body exhibit a tree topology. Therefore, patial information can be assigned to individual parts and the overall information can still be retrived from the end. Angles between a particular keypoint and its parent is calculated by the following formula:

$$
\theta_k = \arctan \left(\frac{y_k -y_{Pa(k)}}{x_k -x_{Pa(k)}} \right)
$$

Where $k$ denotes a particular keypoint other than the "root" of the keypoint tree (the nose) and $Pa(k)$ denotes the parent keypoint of $k$. An average of the confidence rating $c$ between $k$ and $Pa(k)$ is taken to form a new 2-tuple $(\theta, c)$ for all 24 of the non-root keypoints, which contains 48 numbers in total.

\subsection{Architecture}

\subsubsection{Models}

Since we are starting with testing the hypothesis that using keypoint estimation is a feasible approach as opposed to convolutional filtering, we begin with na\"ive model architectures. To begin with, we are going to work with three different feature sets: estimated keypoints, invariant form and a concatenation of the two. Each of these feature sets will correspond to a neural network that has the same structure except for the input layer. 

The BLACkp network takes in the estimated keypoints, the BLACangle network takes in the invariant form of the keypoints, and the BLACkpangle takes in the concatenation of the inputs to the BLACkp and the BLACangle networks. All three BLAC* networks have two hidden fully-connected layers of size 200 and then 128 neurons, and the output is 4 neurons. All the activation functions being used in the models are Rectified Linear Units (ReLU) with bias terms enabled at every layer.

%\begin{figure}[h]
%\centering
%\includegraphics[scale=0.25]{kpNetwork}
%\caption{The keypoints network (BLACkp)}
%\includegraphics[scale=0.25]{angleNetwork}
%\caption{The invariant form network (BLACangle)}
%\includegraphics[scale=0.25]{mixedNetwork}
%\caption{The mixed network (BLACkpangle)}
%\end{figure}

\subsubsection{Training}

-hardware
-regularization or not
-1000 epochs

\section{Experiments}

\begin{figure}[h]
\centering
\includegraphics[scale=0.5]{te_acc}
\end{figure}

\section{Conclusion}

%discuss results
%proves that GAN generation will be more informed when generating a sad person means generating a sad skeleton before generating a whole image
%
%\subsection{Future Work}
%
%Due to limited time we were not able to run cross validation as well as finetuning tradeoff parameters for L1 and L2 regularization.
%
%using a generalized linear model to deal with multiple persons instead of a simple sum
%
%fitting LSTM to generalize into videos
%
%accounting for Confidence in body keypoint measurements dynamically
%
%maybe should use different clases

\section{Contributions}

As with most group projects, each author of this paper contributed a considerable amount of work towards piecing together the project. Jack oversaw the project by organizing and delegating tasks for everyone as well as being the main composer of the paper and poster \footnote{The poster can be found here: \url{https://www.researchgate.net/project/BLACKE}}. Jack and James formulated the theoretical groundwork in the project, Jack and Insoo then went forward with implementing the model and setting up the training environment.

A considerable proportion of the effort went into this project belonged to piecing together a dataset that we can work with. We had the blessing of the Brain and Emotion laboratory in Maastricht University to use their datasets as part of our dataset \citep{de2011bodily}\citep{stienen2012computational}. Aside from the BEAST dataset, Eric and James took the lead in finding images online to include to the dataset and manually hand-labelling them where necessary. After the datset was complete, Gurkaran worked with OpenPose to extract keypoint information from the dataset as well as formulating and implementing the invariant forms for each datapoint. Insoo worked on taking the keypoint data and reorganizing it into a csv file that the models can take in. Eric and Karan worked on a live demo for the presentation.

We would also like to thank Professor Angelica Lim for consultations and guidance toward the design and theoretical groundwork for the project.

\bibliography{references.bib}
\bibliographystyle{plain}

\end{document}
